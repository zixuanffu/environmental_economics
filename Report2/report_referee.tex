\documentclass[12pt]{article}[margin=1in]
\usepackage{setspace}
\linespread{1}
\usepackage{fullpage,graphicx,psfrag,amsmath,amsfonts,verbatim}
\usepackage[small,bf]{caption}
\usepackage{amsthm}
% \usepackage[hidelinks]{hyperref}
\usepackage{hyperref}
\usepackage{bbm} % for the indicator function to look good
\usepackage{color}
\usepackage{mathtools}
\usepackage{fancyhdr} % for the header
\usepackage{booktabs} % for regression table display (toprule, midrule, bottomrule)
\usepackage{adjustbox} % for regression table display
\usepackage{threeparttable} % to use table notes
\usepackage{natbib} % for bibliography
\usepackage{tikz}
\usetikzlibrary{arrows.meta}
\input newcommand.tex
\bibliographystyle{apalike}
% \setlength{\parindent}{0pt} % remove the automatic indentation

\title{\textsc{Adverse Selection in Carbon Offset Markets: Evidence from the Clean Development Mechanism
in China}}
\author{Fu Zixuan}
\date{\today}

\begin{document}
\maketitle
% \thispagestyle{empty}
% \begin{abstract}

% \end{abstract}

% \newpage
% \thispagestyle{empty}
% \tableofcontents
% \newpage

% \setcounter{page}{1}




\section*{1. Overview of the Offset Market Setup}

This paper is informative and enjoyable to read. It offers valuable insights into the \textit{supply side} of the carbon-offset market, focusing on the question: \textit{How are offset products generated by firms?} In particular, the paper examines manufacturing firms in China that invest in abatement projects to reduce emissions from production. These activities generate offset credits (upon approval by an authority), which can then be sold to other regulated firms (often in stricter jurisdictions) to meet their pollution control obligations.

\begin{itemize}
    \item The offset market discussed is \textit{artificially created} by international agreements (e.g., the Kyoto Protocol or Paris Accord). Without such agreements, there would be no demand or supply for offsets.
    \item A third-party entity (the ``market maker'' or \textit{CDM Board}) is responsible for maintaining offset \textit{quality control}. This involves certifying only those projects deemed “additional,” i.e., projects that would not occur without offset revenue.
    \item Economically, this means each offset unit should have a non-zero marginal cost of production. Otherwise, such projects would be profitable without the offset, and hence do not represent real reductions in emissions.
\end{itemize}

The Board's screening process thus aims to approve only those projects with real abatement costs, ensuring \textit{additionality}. The paper models firms' decisions to invest in and apply for offset projects jointly with the Board's approval decision.

\section*{2. The Core Paradox: Registered Firms Have Higher Emissions}

A surprising empirical finding is that firms with approved CDM projects tend to emit \textit{more} overall compared to similar firms without such projects. At first glance, one might expect abatement projects to lower emissions. The authors propose three channels that resolve this paradox:

\begin{enumerate}
    \item \textbf{Growth Effect (Self-Selection):} Firms may invest in abatement for two reasons:
          \begin{itemize}
              \item They have low-cost, high-return opportunities in pollution reduction.
              \item They expect high future output and thus anticipate future returns to justify medium-cost abatement investments.
          \end{itemize}

    \item \textbf{Growth Effect (External Screening):} The CDM Board screens out low-cost projects but does not filter based on firms’ future growth expectations. As a result, registered projects often come from firms with medium abatement costs and high expected growth.

    \item \textbf{Productivity Effect:} Some abatement projects improve emissions efficiency and raise overall productivity. Higher productivity leads firms to increase production, possibly resulting in higher total emissions despite improved emissions intensity.
\end{enumerate}

\section*{3. Comments}

\subsection*{Empirical Results}

\begin{itemize}
    \item The treatment effect estimation is robust. However, it would benefit from more discussion on the construction of control variables (e.g., how distance is computed) and a graphical illustration of control validity.
    \item The authors warn against interpreting the regression results causally due to firm anticipation. But is ``no anticipation'' equivalent to ``no self-selection''? These appear to be distinct assumptions.

          \textit{“Firms may select into the CDM based on their own anticipated growth, violating the ‘no anticipation’ assumption required to interpret an event-study estimate as causal.”}

    \item The observed effect is a mix of:
          \begin{enumerate}
              \item true emissions efficiency improvement,
              \item growth effect (selection),
              \item productivity effect.
          \end{enumerate}
          A valid treatment effect estimation would isolate (1) using a good control group. The current results admit contamination from (2) and (3), but how to isolate (1) remains open. The model primarily separates (2) and (3).
\end{itemize}

\subsection*{Model}

The model is straightforward and effectively captures the core mechanisms: self-selection, external screening, and productivity changes.

\begin{itemize}
    \item Its goal is to disentangle forces (2) and (3), with (2) further divided into different selection types. A key takeaway is the figure (below), which clearly illustrates firms’ investment and application decisions, and the impact of the Board’s screening on these choices.
    \item The model assumes that abatement projects increase emissions efficiency ($z^e$), i.e., force (1) is positive. Yet, whether all such investments actually improve $z^e$ is an empirical question not fully addressed in the paper.
    \item What is the Board’s objective? Presumably, it seeks to maximize the proportion of additional projects among those approved. Given asymmetric information, the Board relies on declared investment costs (or internal rates of return). The model shows that varying the approval threshold doesn’t affect the share of additionality but only the number of approved projects.

          \textbf{Question:} Might the Board also want to maximize the total number of projects, even at the expense of some non-additionality? How can it balance these goals? What instruments does it have? Price of CERs is one, though not directly controlled by the Board. Threshold approval cost is another.
    \item Comparative statics on: (1) changing approval cost thresholds, and (2) CER price changes, would be insightful. Intuitively:
          \begin{itemize}
              \item Lowering the threshold increases participation but may reduce additionality.
              \item Higher CER prices may encourage more genuine abatement.
          \end{itemize}
\end{itemize}

\subsection*{Estimation}

\begin{itemize}
    \item Identification is presented graphically, which may be insufficiently rigorous. But given the challenges of identification in moment-based methods, this is understandable.
    \item Parameter estimates suggest that the paradox primarily stems from selection (force (2)) rather than productivity (force (3)). Which parameter drives this conclusion?
\end{itemize}

\section*{4. Conclusion}

The notion of selection operates on two levels:

\begin{itemize}
    \item \textbf{Adverse Selection:} While the Board may observe baseline emissions and outputs, it lacks access to firms’ private growth expectations. This asymmetry limits its ability to screen for truly additional projects.
    \item \textbf{Selection in Treatment Estimation:} From an econometric perspective, both self-selection into treatment and external screening by the Board pose challenges.
\end{itemize}

The empirical section acknowledges that estimates are contaminated by selection but does not resolve it. The model, by contrast, explores this contamination and identifies the distinct roles of selection and productivity.

Finally, it may be valuable to link this analysis to broader literature on adverse selection and screening. What mechanisms could help the Board achieve first- or second-best outcomes, given these information constraints?


\pagebreak \newpage \bibliography{../References/ref.bib}

\end{document}