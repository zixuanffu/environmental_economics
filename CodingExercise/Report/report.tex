\documentclass[12pt]{article}[margin=1in]
\usepackage{fullpage,graphicx,psfrag,amsmath,amsfonts,verbatim}
\usepackage{multicol,multirow}
\usepackage[small,bf]{caption}
\usepackage{amsthm}
\usepackage{hyperref}
\usepackage{bbm} % for the indicator function to look good
\usepackage{color}
\usepackage{mathtools}
\usepackage{fancyhdr} % for the header
\usepackage{booktabs} % for regression table display (toprule, midrule, bottomrule)
\usepackage{adjustbox} % for regression table display
\usepackage{threeparttable} % to use table notes
\usepackage{natbib} % for bibliography
\input newcommand.tex
\bibliographystyle{apalike}
\setlength{\parindent}{0pt} % remove the automatic indentation % for problem set
% \renewcommand{\thesection}{Question \arabic{section}}
% \renewcommand{\thesubsection}{\arabic{section}.\arabic{subsection}}

% Settings for page number in the footer
\pagestyle{fancy}
\fancyhf{}
\fancyfoot[C]{\thepage}
\renewcommand{\headrulewidth}{0pt}
\renewcommand{\footrulewidth}{0pt}

\title{\textbf{Linear Regression Equation from DDC} \\
\vspace{.3cm}
\large Coding Exercise \\
Environmental Economics}
\author{Zixuan}
\date{\today}

\begin{document}
\maketitle

\setcounter{page}{1}
\section{Preliminaries}
The dynamic discrete choices framework is ubiquitous. However, there seems not to be a standard way to name the different forms of value functions essential to this framework. Therefore, in this section, I will set the stage by defining the set of value functions before proceeding to deriving the so-called euler linear equaiton. 

\paragraph{Reward function}
The variables in the reward functions include state $s$, action $a$, and possibly shock $\epsilon$. 
\text{Total} refers to the inclusion of shock $\epsilon$ while \text{Continuation} refers to the inclusion of future value.
\begin{enumerate}
    \item reward function $u(s, a)$: where $s$ is the state and $a$ is the action.
    \item total reward function $v(s, a, \epsilon)$: where $\epsilon$ is the shock.
    \item continuation reward function $\tilde{u}(s, a) = u(s, a) +  \beta E_{(s_{t+1},\epsilon_{t+1})\mid (a, s, \epsilon)} V(s_{t+1},\epsilon_{t+1})$.
    \item total continuation value function $\tilde{v}(s, a , \epsilon)  = \tilde{u}(s, a) + \epsilon_a$
\end{enumerate}

\paragraph{Value function}
The variables in the value functions inclue state $s$, possibly shock $\epsilon$. 
\begin{enumerate}
    \item value function $V(s, \epsilon) = \max_{a} \tilde{v}(s, a , \epsilon)$
    \item expected (total) value function $$\bar{V}(s) = \int_{\epsilon} V(s, \epsilon) dF(\epsilon)$$
\end{enumerate}


\paragraph{Independence assumption between the shock $\epsilon_a$ across choice } Sometimes I forget that there is a random shock for \textbf{each} choice $a$. They are assumed to be independent across choices, and usually follow a type I extreme value distribution.
\paragraph{Conditional indenpendce assumption between shock $\epsilon_a$ across time} This is an assumption specific to dynamic setting (the first one is assumed in static as well). 
That the future shock is independent of current shock, state and choice.
$$f((s_{t+1}, \epsilon_{t+1})\mid (a, s, \epsilon)) = f(\epsilon_{t+1})f(s_{t+1}\mid (a, s))$$

\paragraph{Relationship}
The relationship between the value function and expected value function therefore can be written as
$$ \bar{V}(s) = \int_{\epsilon} \left [\underbrace{\underbrace{
                \max_{a}u(s, a) + \beta  E_{s_{t+1}\mid (a, s)} \bar{V}(s_{t+1})}_{\tilde{u}(s,a)} + \epsilon_a}_{\tilde{v}(s,a,\epsilon)}\right] dF(\epsilon)$$

which is used to sovle for the expected value function by solving for the fixed point of the above operator.

\section{Linear Regression Equation}

Here the setting is about the supply side of crop. More specifically the the chocie of land use (non-crop, crop etc.)

Recall that in the demand side setting, we have also used discrete choice framework. 
\begin{enumerate}
    \item Individual data: derive a conditional choice probability for individual. Then estimate by MLE. $$\sum \log P(a_i \mid s_i, \theta)$$
    \item Market share data: in the simplest case, everyone $i$ is the same, thus the ccp of each individual $i$ is the market share. then estimate the \textbf{linear equation}.
\end{enumerate}

Here the paper is doing something similar. In the simplest case, every land $i$ is the same.The ccp of each unit $i$ is estimated from the \textit{market share}, which is just the frequency of choosing $j$ (conditioned on the some state variables). 
\subsection{Static case} Recall in the static case, the linear equation is straightforward, 
$$\Pr(a_i = j) = \frac{e^{\alpha_j + \beta x_j + \xi_j}}{1+\sum_{j'} e^{\alpha_{j'} + \beta x_{j'}+\xi_{j'}}}$$
$$\Pr(a_i = 0) = \frac{1}{1+\sum_{j'} e^{\alpha_{j'} + \beta x_{j'}+\xi_{j'}}}$$

Then we can take the log of both sides and take differences,
which gives us the linear regression equation
$$ \log(p_j) -\log(p_0) = \alpha_j + \beta x_j + \xi_j$$

\begin{remark}
    The outside option here in the static case serves a similar purpose as the renewal action in the dynamic case. 
\end{remark}

Treating every $i$ as the same, we can estimate $p_j=\Pr (a_i = j)$ by the fequency of choosing $j$ in the data, which is the \textit{market share} $s_j$ in the demand setting.  

\subsection{Dynamic case} In the dynamic case, do we have a similar linear regression equation? The answer is yes, which is the main contribution of the paper. The following derives it step by step.


In the static case, we have the reward function $u(s,j)$. Then the ccp is just 
$$\Pr(a = j ) = \frac{e^{u(s, j)}}{1+\sum_{a'} e^{u(s, a')}}$$

In the dynamic case, do we have something similar? Yes, but much more complicated.

The ccp in the dynamic case is
$$\Pr_t(j \mid s) = \frac{e^{\tilde{u}_t(s, j)}}{1+\sum_{a'} e^{\tilde{u}_t(s, a')}}$$

where $\tilde{u}_t(s, a) = u_t(s, a) + \beta E_{(s_{t+1},\epsilon_{t+1})\mid (a, s, \epsilon)} V_{t+1}(s_{t+1},\epsilon_{t+1})$.

Take two actions $j$ and $a$ as well and divide and take the log, similar to what we do in the static case.
$$\log \frac{\Pr_t(j \mid s)}{\Pr_t(a \mid s)} = \tilde{u}_t(s, j) - \tilde{u}_t(s, a)$$
Yet this $\tilde{u}_t(s, a)$ is not linear in the parameters we want to estimate straightforwardly due to expectation terms. That is 
$$ \log \frac{\Pr_t(j \mid s)}{\Pr_t(a \mid s)} = u_t(s, j) -u_t(s, a) + \beta E_{(s_{t+1},\epsilon_{t+1})\mid (j, s, \epsilon)} V_{t+1}(s_{t+1},\epsilon_{t+1}) - \beta E_{(s_{t+1},\epsilon_{t+1})\mid (a, s, \epsilon)} V_{t+1}(s_{t+1},\epsilon_{t+1})$$

The only thing left to do now is to find a way to rewrite the expectation term such that they are linear in parameters.

\begin{enumerate}
    \item Thanks to conditional independence assumption,
          $$E_{(s_{t+1},\epsilon_{t+1})\mid (j, s,\epsilon)} V_{t+1}(s_{t+1},\epsilon_{t+1})  = E_{(s_{t+1})\mid (j, s)} \bar{V}(s_{t+1})$$
    \item Let us pick a state $s_{t+1}= s'$ and rewrite the expectation term as
          $$ E_{(s_{t+1})\mid (j, s_t)} \bar{V}_{t+1}(s_{t+1}) = \bar{V}_{t+1}(s_{t+1}=s') + \varepsilon_t(s', j, s)$$
          where we define 
          $$ \varepsilon_t(s', j, s) = E_{(s_{t+1})\mid (j, s)} \bar{V}_{t+1}(s_{t+1}) - \bar{V}_{t+1}(s')$$
          Note that $s_{t+1}$ is a random variable and $s'$ is a fixed state.
    \item Then the foucs is now on $\bar{V}_{t+1}(s_{t+1}=s')$ How do we write $\bar{V}_{t+1}(s_{t+1}=s')$ linear in parameters?
    \item Notice the nice relationship that
          $$ \bar{V}_{t+1}(s_{t+1}=s') = \tilde{u}_{t+1}(s_{t+1}=s', r) - \ln \Pr_{t+1}(r \mid s_{t+1}=s')+\gamma$$
          $$ = u_{t+1}(s', r) + \beta E_{s_{t+2}\mid (r, s_{t+1}=s')} \bar{V}_{t+2}(s_{t+2}) - \ln \Pr_{t+1}(r \mid  s_{t+1}=s') + \gamma$$
    \item This $r$ is not just any action, it is a renewal action that resets things. What does it mean? Let us look what happens at $t+2$ after we take the renewal action at $t+1$.
          $$ E_{s_{t+2}\mid (a^*, s_{t+1}=s')} \bar{V}_{t+2}(s_{t+2}) $$
          $$\int_{s_{t+2}}  \bar{V}_{t+2}(s_{t+2}) d F_{s_{t+2}\mid (r, s_{t+1}=s')}(s_{t+2}) $$
          Therefore, for a different value of $s_{t+1}=s''$, we have different $F(s_{t+2}\mid (r, s_{t+1}=s''))$.
          What renewal action does is to make this the same, that is when choosing the renewal action $r$ at $t+1$, we have
          $$ F(s_{t+2}\mid (r, s_{t+1}=s')) = F(s_{t+2}\mid (r, s_{t+1}=s'')) \forall s_{t+1}=s''$$
    \item Therefore, The difference between two $\bar{V}(s_{t+1}=s')$ and $\bar{V}(s_{t+1}=s'')$ is \textit{clean}.
          $$ \bar{V}(s_{t+1}=s') - \bar{V}(s_{t+1}=s'') = u_{t+1}(s', r) - u_{t+1}(s'', r) - \ln \Pr_{t+1}(r \mid s_{t+1}=s') + \ln \Pr_{t+1}(r \mid s_{t+1}=s'') + $$
          
          
\end{enumerate}

Maybe it's time to write everything down altogether.
\begin{equation}
    \begin{split}
        \log \frac{\Pr_t(j \mid s_t=s)}{\Pr_t(a \mid s_t=s)} & = u_t(s, j) -u_t(s, a) + \beta(\bar{V}_{t+1}(s_{t+1}=s') - \bar{V}_{t+1}(s_{t+1}=s'') )+ \varepsilon_t(s', j, s) - \varepsilon_t(s'', a, s) \\
                                                             & = u_t(s, j) -u_t(s, a)+ \beta  [ u_{t+1}(s', r) - u_{t+1}(s'', r)                                                                           \\&- \ln \Pr_{t+1}(r \mid s_{t+1}=s') + \ln \Pr_{t+1}(r \mid s_{t+1}=s'')]
        + \varepsilon_t(s', j, s) - \varepsilon_t(s'', a, s)                                                                                                                                               \\
    \end{split}
\end{equation}
Rearranging the terms, we have
\begin{equation}
    \begin{split}
        \log \frac{\Pr_t(j \mid s_t=s)}{\Pr_t(a \mid s_t=s)} + \beta(\frac{\Pr_{t+1}(r \mid s_{t+1}=s')}{\Pr_{t+1}(r \mid s_{t+1}=s'')}) & =u_t(s, j) -u_t(s, a)                                                                                                            \\ 
                                                                                                                                         & + \beta \left [ u_{t+1}(s', r) - u_{t+1}(s'', r)                   \right ] + \varepsilon_t(s', j, s) - \varepsilon_t(s'', a, s) \\
    \end{split}
\end{equation}

Hmmm, this linear equation looks cool but the $\varepsilon_t$ looks really suspicious to me here. Also, it seems to be calling for lots of assumptions on the form of $u(s, a)$. 

Let us delve into them.

\paragraph{Reward function $u(s, a)$}
As a function of state $s$ and action $a$,
$$ u(s, a) = \alpha(s,a) + \alpha_R R(s,a) + \xi(s,a)$$
In the paper,

$$  u(k, a) = \alpha(k,a) + \alpha_R R(a) + \xi(k,a)$$

Also the paper assumes that individual state (let us ignore aggregate state for the moment) evolves deterministically, the individual state is the same as last period's action.

$$ u(k, a) = \theta_k + \theta_{ka} + \alpha_R R(a) + \xi(k,a)$$

Therefore the linear regression equation can be written as
\begin{equation} \label{eq:linear_regression}
    \begin{split}
        \log \frac{\Pr_t(j \mid k)}{\Pr_t(a \mid k)} + \beta\log(\frac{\Pr_{t+1}(r \mid j)}{\Pr_{t+1}(r \mid a)})
         & =  \alpha_{k,j}-\alpha_{k,a}+\beta(\alpha_{j,r}-\alpha_{a,r}) + \alpha_R (R_t(j) - R_t(a)) \\ & + \xi_t(k,j)-\xi_t(k,a) + \beta(\xi_{t+1}(j, r) - \xi_{t+1}(a, r)) + \varepsilon_t(j,k) - \varepsilon_t(a,k)\\
    \end{split}
\end{equation}


\paragraph{Clarification on notation}
\begin{enumerate}
    \item When a variable is indexed by time $t$, it is a random variable, for example $s_{t+1}$.
    \item When a variable is not indexed by $t$, it is a fixed value, for example $s'$. The parameter is also a fixed value, for example $\alpha_{k,j}$. \item When a function is indexed by time $t$, it means the function input needs to be a variable at time $t$, for example you need to put variables from time $t$ into function $u_t$. For the choice probability $\Pr_t$, it is also a function, same idea.
    \item When a function is not indexed by $t$, it just means the function form.
\end{enumerate}

\paragraph{Restriction and normalization}

The parameters we care about are $\alpha_{k,j}$ for all $(k,j)$ pairs and $\alpha_R$. However, for one specific $k$, we can only have one intercept recovered. When indeed we have $\alpha_{k,j}$ and $\alpha_{k,a}$ that we care about. Therefore, we need to make some assumptions. The assumption that this paper makes is that for one choice $k^*$ all the $\alpha_{kk^*}$ is the same for all current state $k$. And we normalize it to $\alpha_{kk^*} = 0$.

The problem set takes a similar approach, for a fixed $k$ and some fixed $jar$, we know the value of
$$ \theta_{j} + \theta_{a} + \theta_{kj} + \theta_{ka} + \beta(\theta_{jr} - \theta_{ar})$$
Then just pick another $k'= a$ and the same fixed $jar$, we know 
$$ \theta_{j} + \theta_{a} + \theta_{aj} + \theta_{aa} + \beta(\theta_{jr} - \theta_{ar})$$
Then we can take the difference between the two equations, which gives us
$$ \theta_{kj} + \theta_{ka} - \theta_{aj} - \theta_{aa} $$
Restrict that for one choice $j^*$, we have $\theta_{kj^*} = 0$ for all $k$. Then the above becomes $$ \theta_{ka} - \theta_{aa}$$
We also assume that $\theta_{aa} = 0$ for all $a$. Then we recovre $ \theta_{ka}$

Therefore to recover $\theta_{ka}$, 
\begin{enumerate}
    \item We need $\Delta_{kj^*ar}$ and $\Delta_{aj^*ar}$
    \item Then $\theta_{ka} = \Delta_{kj^*ar} - \Delta_{aj^*ar}$ if we assume $\theta_{kj^*} = 0$ for all $k$ and $\theta_{aa} = 0$ for all $a$.
\end{enumerate}

\subsection{Estimation}
We construct a panel data. The unit is indexed by $k$ and $j$ and $a$ and $r$ (and $m$). The time is indexed by $t$. Then we adopt techniques in panel data estimation depending on the assumptions we want to make. 


\subsection{Results}


\end{document}


